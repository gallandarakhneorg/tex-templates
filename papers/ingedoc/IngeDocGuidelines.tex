\documentclass{ingedoc}

\usepackage[utf8]{inputenc}
\usepackage{graphicx}
\usepackage{tabularx}

\newcommand{\ingedocs}{IngéDoc-2012\xspace}

\title{Author Guidelines for \ingedocs}

\author{Gillian Basso\inst{1}\email{gillian.basso@utbm.fr} \and Stéphane Galland\inst{1}}

\affiliation{
IRTES-SET \\
Université de Technologie de Belfort-Montbéliard
}

% The abstract must be defined in the PREAMBLE
\begin{abstract}
This document describes the guidelines for the \ingedocs Conference. This document provides also a simple example of how a paper may be written.
\end{abstract}

% The keywords must be defined in the PREAMBLE
\keywords{IngéDoc, Author Guidelines}

\thanks{Thanks to UTBM.}

\begin{document}

This document describes the \LaTeX\ style for the \ingedocs Conference.
The installation of the style is explained. The configuration of the style is briefly explained according to the step in the submission process.

\section{Style Package Preparation}

\textcolor{red}{This section is for the Conference Chairs, not for the Authors. It is assumed that the conference chairs are running the Linux operating system.}

This section is dedicated to the members of the Organization Board. A complete \LaTeX\ distribution must be already installed on your system.

The packaging of the \ingedocs \LaTeX\ style from a Linux operating system is:
\begin{enumerate}
\item Download \texttt{ingedoc-src-2012.tar.gz}
\item Unpack \texttt{ingedoc-src-2012.tar.gz}
\item Open an interactive shell
\item \texttt{cd path/to/ingedocs-2012}
\item \texttt{cd adobe-fonts}
\item Run: \texttt{./generatefonts.sh}
\item \texttt{cd ..}
\item \texttt{rm -rf adobe-fonts}
\item \texttt{cd ..}
\item \texttt{tar cvfz ingedocs-for-authors-2012.tar.gz path/to/ingedocs-2012}
\item Provide the file \texttt{ingedocs-for-authors-2012.tar.gz} to the authors.
\end{enumerate}

\section{Installation for Authors on Linux}

This section is dedicated to the installation of the \LaTeX\ style of \ingedocs for the authors on the Linux operating system.

\begin{enumerate}
\item Download \texttt{ingedoc-for-authors-2012.tar.gz}
\item Unpack \texttt{ingedoc-for-authors-2012.tar.gz}
\item Open an interactive shell
\item \texttt{cd path/to/ingedocs-2012}
\item Run: \texttt{./installfonts.sh}
\end{enumerate}

\section{Installation for Authors on Windows}

This section is dedicated to the installation of the \LaTeX\ style of \ingedocs for the authors on the Windows operating system.

\begin{enumerate}
\item Download \texttt{ingedoc-for-authors-2012.tar.gz}
\item Unpack \texttt{ingedoc-for-authors-2012.tar.gz}
\item Go to the folder \texttt{path{\textbackslash}to{\textbackslash}ingedocs-2012}
\item Run \texttt{installfonts.bat} by double-clicking on it.
\end{enumerate}

\section{Uninstallation for Authors on Linux}

This section is dedicated to the uninstallation of the \LaTeX\ style of \ingedocs for the authors on the Linux operating system.

\begin{enumerate}
\item Open an interactive shell
\item \texttt{cd path/to/ingedocs-2012}
\item Run: \texttt{./uninstallfonts.sh}
\end{enumerate}

\section{Uninstallation for Authors on Windows}

This section is dedicated to the uninstallation of the \LaTeX\ style of \ingedocs for the authors on the Windows operating system.

\begin{enumerate}
\item Go to the folder \texttt{path{\textbackslash}to{\textbackslash}ingedocs-2012}
\item Run \texttt{uninstallfonts.bat} by double-clicking on it.
\end{enumerate}

\section{Configuration of the \LaTeX\ Class}

Now, you are able to write your paper for \ingedocs.
The \ingedocs guuidelines for the authors are all coded in the \LaTeX\ class \texttt{ingedoc}.
For writing your paper for \ingedocs, you must use this document class, as illustrated by the following skeleton:
\begin{verbatim}
\documentclass[options]{ingedoc}
% The preamble of your document
\begin{document}
% The text of your paper
\end{document}
\end{verbatim}

The \texttt{options} of the \texttt{ingedoc} class are described in the following subsections.

The preamble of the \LaTeX\ document must contains the following elements:  title, authors, affiliation, abstract, keywords, and acknowledgements. See the following sections for details.

\subsection{Class Options}

\subsubsection{Language Definition}
	\begin{itemize}
	\item \texttt{english}: the paper is written in English. Default option.
	\item \texttt{french}: the paper is written in French.
	\end{itemize}
\subsubsection{Submission Stage}
	\begin{itemize}
	\item \texttt{draft}: the paper is a draft: the names of the authors are not blinded, only the figures' bounds are output, and ``The Sans'' font is not used (note that the used fonts are close to ``The Sans'').
	\item \texttt{submit}: the paper is written in a submittable version: the names of the authors are not blinded, the figures are output, and ``The Sans'' font is not used (note that the used fonts are close to ``The Sans''). Default option.
	\item \texttt{blind}: the paper is written in a version for the reviewers: the names of the authors are blinded, the figures are output, and ``The Sans'' font is not used (note that the used fonts are close to ``The Sans'').
	\item \texttt{final}: the paper is ready to be printed in the proceedings: the names of the authors are not blinded, the figures are output, and the ``The Sans'' font is used.
	\end{itemize}
\subsubsection{Lettrine}
	\begin{itemize}
	\item \texttt{lettrine}: the paper starts with a lettrine. Default option.
	\item \texttt{nolettrine}: the paper does not start with a lettrine.
	\end{itemize}
\subsubsection{Non-Free Font}
	\begin{itemize}
	\item \texttt{thesansfont}: the proprietary/non-free font \texttt{TheSans} must be used.
	\item \texttt{nothesansfont}: the proprietary/non-free font \texttt{TheSans} must not used.
	\end{itemize}

\subsection{Title of the Paper}

The title of the paper must be defined in the preamble of the \LaTeX\ document with the \texttt{{\textbackslash}title} macro.

\textthesansbold{Syntax:} \texttt{{\textbackslash}title\{the title of the paper\}}

\textthesansbold{Example:} \texttt{{\textbackslash}title\{Author Guidelines for \ingedocs\}}

\subsection{Authors of the Paper}

The authors of the paper must be defined in the preamble of the \LaTeX\ document with the \texttt{{\textbackslash}author} macro.

\textthesansbold{Syntax:} \texttt{{\textbackslash}author\{the authors\}}

The names of the authors must be separated by the \texttt{{\textbackslash}and} macro.

Each name could be following by one of the following macros:
\begin{itemize}
\item \texttt{{\textbackslash}email\{adr@domain.com\}}: permits to specify the contact email of the author.
\item \texttt{{\textbackslash}inst\{number\}}: permits to specify the institution of the author. The institution is specified in the affiliation macro (see Section~\ref{sec:affiliation}). The number is associated to each institution by the affiliation macro.
\end{itemize}

\textthesansbold{Example:} \texttt{{\textbackslash}author\{Gillian Basso{\textbackslash}inst\{1\} {\textbackslash}email\{gillian.basso@utbm.fr\} {\textbackslash}and Stéphane Galland{\textbackslash}inst\{12\}\}}

\subsection{Affiliation of the Authors}\label{sec:affiliation}

The affiliation of the authors must be defined in the preamble of the \LaTeX\ document with the \texttt{{\textbackslash}affiliation} macro.

\textthesansbold{Syntax:} \texttt{{\textbackslash}affiliation\{institutions\}}


The institutions are separated by the macro \texttt{{\textbackslash}and}.

\textthesansbold{Example:} \texttt{{\textbackslash}affiliation\{
IRTES-SET {\textbackslash}{\textbackslash}
Université de Technologie de Belfort-Montbéliard
{\textbackslash}and
CITAT {\textbackslash}{\textbackslash}
Universidad de Technologia de Tucum{\textbackslash}`an
\}}

\subsection{Abstract of the Paper}

The abstract of the paper must be defined in the preamble of the \LaTeX\ document inside the environment \texttt{abstract}.

\textthesansbold{Syntax:} \texttt{{\textbackslash}begin\{abstract\} text of the abstract {\textbackslash}end\{abstract\}}

\textthesansbold{Example:} \texttt{{\textbackslash}begin\{abstract\} This document describes the guidelines for the \ingedocs Conference. This document provides also a simple example of how a paper may be written. {\textbackslash}end\{abstract\}}

\subsection{Keywords of the Paper}

The keywords of the paper must be defined in the preamble of the \LaTeX\ document with the \texttt{{\textbackslash}keywords} macro.

\textthesansbold{Syntax:} \texttt{{\textbackslash}keywords\{ the keywords \}}

\textthesansbold{Example:} \texttt{{\textbackslash}keywords\{ IngéDoc, Author Guidelines \}}

\subsection{Bibliography}

To include a bibliography, you should write a Bib\TeX\ file, and include it with the \texttt{{\textbackslash}bibliography} macro.

\textthesansbold{Example:} \texttt{{\textbackslash}bibliography\{biblio\}}

\subsection{Acknowledgement}

You are able to put acknowledgements in your paper with the \texttt{{\textbackslash}thanks} macro.

\textthesansbold{Syntax:} \texttt{{\textbackslash}thanks\{ text \}}

\textthesansbold{Example:} \texttt{{\textbackslash}thanks\{ Thanks to UTBM. \}}

\section{Additional Guidelines}

This section details the author guidelines that are not directly supported by the \LaTeX\ style.

\subsection{Figures}

Each figure must have a caption, as illustrated by Figure~\ref{fig:thefigure}.

\begin{verbatim}
\begin{figure}
	\includegraphics{utbm.pdf}
	\caption{Example of a Figure}
	\label{fig:thefigure}
\end{figure}
\end{verbatim}

\begin{figure}
	\includegraphics{utbm.pdf}
	\caption{Example of a Figure}
	\label{fig:thefigure}
\end{figure}


The preferred formats for the images are (in the preference order):
\begin{itemize}
\item \texttt{.pdf} with embedded vectorial picture.
\item \texttt{.png}
\item \texttt{.jpeg}
\end{itemize}

\subsection{Tables}

All the tables and arrays must be put inside a \texttt{table} float.
Each table must have a caption, as illustrated by Table~\ref{tab:thetable}.
If you table has columns with titles, each title must be formatted with the \texttt{{\textbackslash}tabletitle} macro.

\begin{verbatim}
\begin{table}
	\begin{tabular}{|l|l|l|}
	\hline
	\tabletitle{H 1} & \tabletitle{H 2} &
	\tabletitle{H 3} \\
	\hline
	Cell 1 & Cell 2 & Cell 2 \\
	\hline
	\end{tabular}
	\caption{Example of a Figure}
	\label{tab:thetable}
\end{table}
\end{verbatim}

\begin{table}
	\begin{tabular}{|l|l|l|}
	\hline
	\tabletitle{H 1} & \tabletitle{H 2} & \tabletitle{H 3} \\
	\hline
	Cell 1 & Cell 2 & Cell 2 \\
	\hline
	\end{tabular}
	\caption{Example of a Figure}
	\label{tab:thetable}
\end{table}

\subsection{Style of the Text}

The \ingedocs class redefines the standard \LaTeX\ macros related to the text style:
\begin{itemize}
\item \texttt{{\textbackslash}emph\{text\}} is emphazing the given text; example: \emph{Emphazed Text}.
\item \texttt{{\textbackslash}textbf\{text\}} is output the given text with a bold face; example: \textbf{Bold Text}. The macro \texttt{{\textbackslash}emph} should be preferred to the macro \texttt{{\textbackslash}textbf}.
\item \texttt{{\textbackslash}textit\{text\}} is output the given text with an italic face; example: \textit{Italic Text}. The macro \texttt{{\textbackslash}emph} should be preferred to the macro \texttt{{\textbackslash}textit}.
\item \texttt{{\textbackslash}textmd\{text\}} is output the given text with a semi-bold face; example: \textmd{Semi-Bold Text}.
\item \texttt{{\textbackslash}texttt\{text\}} is output the given text with a typewriter font; example: \texttt{Typewriter Text}.
\item \texttt{{\textbackslash}textsc\{text\}} is output the given text with a small-cap font; example: \textsc{Small-Cap Text}.
\item \texttt{{\textbackslash}textup\{text\}} is output the given text with the standard face; example: \textup{Standard Text}.
\end{itemize}


\textbf{Caution:} the \TeX\ macros (\texttt{{\textbackslash}bfseries}, \texttt{{\textbackslash}scshape}...) are not redefined by the \ingedocs style. They are still using the standard \TeX\ fonts and not ``The Sans'' font.

\subsection{Bibliography}

The bibliography style must be \texttt{plain}.

\nocite*
\bibliography{biblio}

\section{Loaded \LaTeX\ Packages}

The \ingedocs class is loaded the following \LaTeX\ packages, so that you could use them in your paper:
\begin{itemize}
\item \texttt{ifthen}
\item \texttt{mathpazo}
\item \texttt{fontenc}
\item \texttt{lettrine}
\item \texttt{vmargin}
\item \texttt{color}
\item \texttt{xstring}
\item \texttt{enumitem}
\item \texttt{eso-pic}
\item \texttt{picture}
\item \texttt{xcolor}
\item \texttt{geometry}
\item \texttt{xspace}
\end{itemize}

\end{document}
