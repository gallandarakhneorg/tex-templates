%% Use the standard UP-methodology class
%% with French language.
%%
%% You may specify the option 'twoside' or 'oneside' for
%% the document.
%%
%% See the documentation tex-upmethodology on
%% http://www.arakhne.org/tex-upmethodology/
%% for details about the macros that are provided by the class and
%% to obtain the list of the packages that are already included. 
 
\documentclass[english]{spimubphdthesis}
 
%%--------------------
%% The TeX code is entering with UTF8
%% character encoding (Linux and MacOS standards)
\usepackage[utf8]{inputenc}
 
%%-------------------
%% NATBIB is enabled by default.
%% If you would like to disable it, put the 'nonatbib' option to the class' options.
 
%%--------------------
%% Include the 'multibib' package to enable to
%% have different types of bibliographies in the
%% document (see at the end of this template for
%% an example with a personnal bibliography and
%% a general bibliography)
%%
%% Each bibliography defined with 'multibib'
%% adds a chapter with the corresponding
%% publications (in addition to the chapter for
%% the standard/general bibliography).
%% CAUTION:
%% There is no standard way to do include this type of
%% personnal bibliography.
%% We propose to use 'multibib' package to help you,
%% for example.
%\usepackage{multibib}
 
%% Define a "type" of bibliography, here the PERSONAL one,
%% that is supported by 'multibib'.
%\newcites{PERSO}{Liste de mes publications}
 
%% To cite one of your PERSONAL papers with the style
%% of the PERSONAL bibliography: \citePERSO{key}
%% To force to show one of your PERSONAL papers into
%% the PERSONAL bibliography, even if not cited in the
%% text: \nocitePERSO{key}
 
%% REMARK: When you are using 'multibib', you
%% must compile the PERSONAL bibliography by hand.
%% For example, the sequence of commands to run
%% when you had defined the bibliography PERSO is:
%%   $ pdflatex my_document.tex
%%   $ bibtex my_document.aux
%%   $ bibtex PERSO.aux
%%   $ pdflatex my_document.tex
%%   $ pdflatex my_document.tex
%%   $ pdflatex my_document.tex
 
%%--------------------
%% Add here any other packages that are needed for your document.
%\usepackage{eurosim}
%\usepackage{amsmath}
 
%%--------------------
%% Set the title, subtitle, defense date, and
%% the registration number of the PhD thesis.
%% The optional parameter is the subtitle of the PhD thesis.
%% The first mandatory parameter is the title of the PhD thesis.
%% The second mandatory parameter is the date of the PhD defense.
%% The third mandatory parameter is the location/city of the PhD defense.
%% The forth mandatory parameter is the reference number given by
%% the University Library after the PhD defense.
\declarethesis[Subtitle]{Title}{17 septembre 2012}{Belfort}{XXX}

%%--------------------
%% Set the title in the secondary language
%% (it is in English if the PhD dissertation is written in French; it is French if the PhD dissertation is written in English)
\declareminorthesistitle{Title in secondary language}

%%--------------------
%% Set the author of the PhD thesis
\addauthor[email]{Firstname}{Name}
 
%%--------------------
%% Add a member of the jury
%%
%% CAUTION 1: If a Jury member is not present during the defense,
%%            she/he must be in the list of the Jury members.
%%            Only the reviewers and the members who are present during the defense must
%%            appear in the Jyry member list. 
%% CAUTION 2: After your defense, you must assign the role "Pr\'esident" to
%%            the Jury member who have been the President of the Jury.
%% CAUTION 3: The recommended order for the Jury members is:
%%            President, Reviewer(s), Examiner(s), Director(s),
%%            Other supervisor(s), Invited person(s).
%% \addjury{Firstname}{Lastname}{Role in the jury}{Position}
\addjury{Incroyable}{Hulk}{Pr\'esident}{Professeur à l'Université de Gotham City \\ Commentaire secondaire}
\addjury{Captain}{America}{Rapporteur}{Professeur à l'Université USA}
\addjury{Super}{Man}{Examinateur}{Professeur à l'Université de Gotham City}
\addjury{Bat}{Man}{Directeur de thèse}{Professeur à l'Université de Gotham City}
\addjury{The}{Volwerine}{Codirecteur de thèse}{Professeur à l'Université de Gotham City}
\addjury{Pac}{Man}{Invité}{Professeur quelque part}
 
%%--------------------
%% Change style of the table of the jury
%% \Set{jurystyle}{put macros for the style}
%\Set{jurystyle}{\small}
 
%%--------------------
%% Set the English abstract
\thesisabstract[english]{This is the abstract in English.}
 
%%--------------------
%% Set the English keywords. They only appear if
%% there is an English abstract
\thesiskeywords[english]{Keyword 1, Keyword 2}
 
%%--------------------
%% Set the French abstract
\thesisabstract[french]{Ceci est le résumé en français.}
 
%%--------------------
%% Set the French keywords. They only appear if
%% there is an French abstract
\thesiskeywords[french]{Mot-cl\'e 1, Mot-cl\'e 2}
 
%%--------------------
%% Change the layout and the style of the text of the "primary" abstract.
%% If your document is written in French, the primary abstract is in French,
%% otherwise it is in English.
%\Set{primaryabstractstyle}{\tiny}
 
%%--------------------
%% Change the layout and the style of the text of the "secondary" abstract.
%% If your document is written in French, the secondary abstract is in English,
%% otherwise it is in French.
%\Set{secondaryabstractstyle}{\tiny}
 
%%--------------------
%% Change the layout and the style of the text of the "primary" keywords.
%% If your document is written in French, the primary keywords are in French,
%% otherwise they are in English.
%\Set{primarykeywordstyle}{\tiny}
 
%%--------------------
%% Change the layout and the style of the text of the "secondary" keywords.
%% If your document is written in French, the secondary keywords are in English,
%% otherwise they are in French.
%\Set{secondarykeywordstyle}{\tiny}
 
%%--------------------
%% Change the speciality of the PhD thesis
%\Set{speciality}{Informatique}
 
%%--------------------
%% Change the institution
%\Set{universityname}{Universit\'e de Technologie de Belfort-Montb\'eliard}
 

%%--------------------
%% Clear the list of the laboratories
\resetlaboratories

%%--------------------
%% Add the laboratory where the thesis was made
\addlaboratory{Laboratoire Connaissance et Intelligence Artificielle Distribu\'ees}

%%--------------------
%% The name of the university that is jointly delivering the Doctoral degree with UBFC
\Set{jointuniversity}{Universit\'e de Ngaound\'er\'e au Cameroun}

%%--------------------
%% Clear the list of the partner/sponsor logos
%\resetpartners

%%--------------------
%% Add the logos of the partners or the sponsors on the front page
%%
%% CAUTION 1: At least, the logo of the University should appear (UB)
%%
%\addpartner[image options]{image name}

%\addpartner{ub}

%%--------------------
%% Change the header and the foot of the pages.
%% You must include the package "fancyhdr" to
%% have access to these macros.
%% Left header
%\lhead{}
%% Center header
%\chead{}
%% Right header
%\rhead{}
%% Left footer
%\lfoot{}
%% Center footer
%\cfoot{}
%% Right footer
%\rfoot{}
 
%%--------------------
% Declare several theorems
\declareupmtheorem{mytheorem}{My Theorem}{List of my Theorems}

%%--------------------
%% Change the message on the backcover.
%\Set{backcovermessage}{%
%	Some text
%}

\begin{document}
 
%%--------------------
%% The following line does nothing until
%% the class option 'nofrontmatter' is given.
%\frontmatter

%%--------------------
%% The following line permits to add a chapter for "acknowledgements"
%% at the beginning of the document. This chapter has not a chapter
%% number (using the "star-ed" version of \chapter) to prevent it to
%% be in the table of contents
\chapter*{Acknowledgements}
 
%%--------------------
%% Include a general table of contents
\tableofcontents

%%--------------------
%% The content of the PhD thesis
\mainmatter
 
\part{Context et Problems}

\chapter{Introduction}
 
\section{Context}

This skeleton of report gives several elements in order to help you to write your PhD dissertation.
A typical outline is provided.

\section{Objectives of the thesis}

The main objective of the thesis could be emphasis with:

\begin{emphbox}
	Propose a model for doing something!
\end{emphbox}

\section{Outline of the PhD thesis dissertation}

\chapter{Context and Problems}

\section{Introduction}

\section{Context}

Describe the context of the PhD thesis, including the definitions.

\section{Problems}

Explain the problems related to the context.
Details the ones that are takled into this dissertation.

\section{Conclusion}

\chapter{State of Art}

% Automatic Lettrine and Minitoc at the begining of the chapter.
%
% The following macros are formatting the text at the begining of a chapter according to the
% standard format.
%
% \chapterintro               See \chapterintrotosection.
%
% \chapterintro*              Similar to \chapterintro, except that the minitoc will be ignored.
%
% \chapterintrotosection      transform to lettrine the first letter that is following this macro
%                             until the next following \section macro.
%                             YOU MUST type the \section macro
%                             in the same file as the \chapterintrotosection macro.
%                             AND
%                             put a minitoc (if the minitoc package is included) just before
%                             the next following \section macro.
%
% \chapterintrotoinput        transform to lettrine the first letter that is following this macro
%                             until the next following \input macro.
%                             YOU MUST type the \input macro
%                             in the same file as the \chapterintrotoinput macro.
%                             AND
%                             put a minitoc (if the minitoc package is included) just before
%                             the next following \input macro.
%
% \chapterintrotoinclude      transform to lettrine the first letter that is following this macro
%                             until the next following \include macro.
%                             YOU MUST type the \include macro
%                             in the same file as the \chapterintrotoinclude macro.
%                             AND
%                             put a minitoc (if the minitoc package is included) just before
%                             the next following \include macro.

%\chapterintro*
%\chapterintro
%\chapterintrotosection
%\chapterintrotoinput
%\chapterintrotoinclude

\section{Introduction}

\section{Detailed state-of-art}

\section{Conclusion}
 
%% Citation from the general bibliography
%\cite{key}
 
%% Citation from the PERSO bibliography
%\citePERSO{key}
 
\part{Contribution}

\chapter{Contribution}

\section{Introduction}

\section{Détails of the contribution}

\section{Conclusion}

\chapter{Application}

\section{Introduction}

\section{Presentation of the application}

\section{Specific implementation of the application}

\section{Experimental results}

\section{Conclusion}

\part{Conclusion}

\chapter{General conclusion}
 
\section{Summary of the PhD thesis}

\section{Perpectives}
 
%%--------------------
%% Start the end of the thesis
\backmatter
 
%%--------------------
%% Bibliography
 
%% PERSONAL BIBLIOGRAPHY (use 'multibib')
 
%% Change the style of the PERSONAL bibliography
%\bibliographystylePERSO{phdthesisapa}
 
%% Add the chapter with the PERSONAL bibliogaphy.
%% The name of the BibTeX file may be the same as
%% the one for the general bibliography.
%\bibliographyPERSO{biblio.bib}
 
%% Below, include a chapter for the GENERAL bibliography.
%% It is assumed that the standard BibTeX tool/approach
%% is used.
 
%% GENERAL BIBLIOGRAPHY
 
%% To cite one of your PERSONAL papers with the style
%% of the PERSONAL bibliography: \cite{key}
 
%% To force to show one of your PERSONAL papers into
%% the PERSONAL bibliography, even if not cited in the
%% text: \nocite{key}
 
%% The following line set the style of
%% the GENERAL bibliogaphy.
%% The following is an adaptation of "dinat.bst".
\bibliographystyle{spimphdthesis}
 
%% Link the GENERAL bibliogaphy to a BibTeX file.
\bibliography{biblio.bib}
 
%%--------------------
%% List of figures and tables
 
%% Include a chapter with a list of all the figures.
%% In French typograhic standard, this list must be at
%% the end of the document.
\listoffigures
 
%% Include a chapter with a list of all the tables.
%% In French typograhic standard, this list must be at
%% the end of the document.
\listoftables
 
%%--------------------
%% Include a list of definitions
\listofdefinitions

%%--------------------
%% Appendixes
\appendix
\part{Appendix}
 
\chapter{First appendix}

\chapter{Second appendix}
 
\end{document}
