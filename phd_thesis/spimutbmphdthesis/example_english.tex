% % Use the standard UP-methodology class
% % with English language.
% %
% % You may specify the option 'twoside' or 'oneside' for
% % the document.
% %
% % See the documentation tex-upmethodology on
% % http://www.arakhne.org/tex-upmethodology/
% % for details about the macros that are provided by the class and
% % to obtain the list of the packages that are already included.

\documentclass[english]{spimutbmphdthesis}

% %--------------------
% % The TeX code is entering with UTF8
% % character encoding (Linux and MacOS standards)
\usepackage[utf8]{inputenc}

% %-------------------
% % NATBIB is enabled by default.
% % If you would like to disable it, put the 'nonatbib' option to the class' options.

% %--------------------
% % Include the 'multibib' package to enable to
% % have different types of bibliographies in the
% % document (see at the end of this template for
% % an example with a personal bibliography and
% % a general bibliography)
% %
% % Each bibliography defined with 'multibib'
% % adds a chapter with the corresponding
% % publications (in addition to the chapter for
% % the standard/general bibliography).
% % CAUTION:
% % There is no standard way to do include this type of
% % personal bibliography.
% % We propose to use 'multibib' package to help you,
% % for example.
%\usepackage{multibib}

% % Define a "type" of bibliography, here the PERSONAL one,
% % that is supported by 'multibib'.
%\newcites{PERSO}{List of my publications}

% % To cite one of your PERSONAL papers with the style
% % of the PERSONAL bibliography: \citePERSO{key}
% % To force to show one of your PERSONAL papers into
% % the PERSONAL bibliography, even if not cited in the
% % text: \nocitePERSO{key}

% % REMARK: When you are using 'multibib', you
% % must compile the PERSONAL bibliography by hand.
% % For example, the sequence of commands to run
% % when you had defined the bibliography PERSO is:
% %   $ pdflatex my_document.tex
% %   $ bibtex my_document.aux
% %   $ bibtex PERSO.aux
% %   $ pdflatex my_document.tex
% %   $ pdflatex my_document.tex
% %   $ pdflatex my_document.tex

% %--------------------
% % Add here any other packages that are needed for your document.
%\usepackage{eurosim}
%\usepackage{amsmath}

% %--------------------
% % Set the title, subtitle, defense date, and
% % the registration number of the PhD thesis.
% % The optional parameter is the subtitle of the PhD thesis.
% % The first mandatory parameter is the title of the PhD thesis.
% % The second mandatory parameter is the date of the PhD defense.
% % The third mandatory parameter is the location/city of the PhD defense.
% % The fourth mandatory parameter is the reference number given by
% % the University Library after the PhD defense.
\declarethesis[Subtitle]{Title}{September 17, 2012}{Belfort}{XXX}

% %--------------------
% % Set the title in the secondary language
% % (it is in English if the PhD dissertation is written in French; it is French if the PhD dissertation is written in English)
\declareminorthesistitle{Titre en francais}

% %--------------------
% % Set the author of the PhD thesis
\addauthor[email]{Firstname}{Lastname}

% %--------------------
% % Add a member of the jury
% %
% % Two macros are provided: one without civility, one with civility.
% %
% % CAUTION 1: If a Jury member is not present during the defense,
% %            she/he must be in the list of the Jury members.
% %            Only the reviewers and the members who are present during the defense must
% %            appear in the Jury member list.
% % CAUTION 2: After your defense, you must assign the role "President" to
% %            the Jury member who has been the President of the Jury.
% % CAUTION 3: The recommended order for the Jury members is:
% %            President, Reviewer(s), Examiner(s), Director(s),
% %            Other supervisor(s), Invited person(s).
% % \addjury{Firstname}{Lastname}{Role in the jury}{Position}
\addjury{Incredible}{Hulk}{President}{Professor at Gotham City University \\ Secondary comment}
\addjury{Captain}{America}{Reviewer}{Professor at USA University}
\addjury{Super}{Man}{Examiner}{Professor at Gotham City University}
% % \addjuryex{Civility}{Firstname}{Lastname}{Role in the jury}{Position}
\addjuryex{Mr.}{Bat}{Man}{Thesis Director}{Professor at Gotham City University}
\addjuryex{Mr.}{The}{Wolverine}{Co-Director}{Professor at Gotham City University}
\addjuryex{Ms.}{Pac}{Man}{Invited}{Professor somewhere}

% %--------------------
% % Change style of the table of the jury
% % \Set{jurystyle}{put macros for the style}
%\Set{jurystyle}{\small}

% %--------------------
% % Set the English abstract
\thesisabstract[english]{This is the abstract in English. This is the abstract in English. This is the abstract in English. This is the abstract in English. This is the abstract in English. This is the abstract in English. This is the abstract in English. This is the abstract in English. This is the abstract in English. This is the abstract in English. This is the abstract in English. This is the abstract in English. This is the abstract in English. This is the abstract in English. This is the abstract in English. This is the abstract in English.}

% %--------------------
% % Set the English keywords. They only appear if
% % there is an English abstract
\thesiskeywords[english]{Keyword 1, Keyword 2}

% %--------------------
% % Set the French abstract
\thesisabstract[french]{Ceci est un résumé en français. Ceci est un résumé en français. Ceci est un résumé en français. Ceci est un résumé en français. Ceci est un résumé en français. Ceci est un résumé en français. Ceci est un résumé en français. Ceci est un résumé en français. Ceci est un résumé en français. Ceci est un résumé en français. Ceci est un résumé en français. Ceci est un résumé en français. Ceci est un résumé en français. Ceci est un résumé en français. Ceci est un résumé en français. Ceci est un résumé en français. Ceci est un résumé en français. Ceci est un résumé en français. Ceci est un résumé en français.}

% %--------------------
% % Set the French keywords. They only appear if
% % there is a French abstract
\thesiskeywords[french]{Mot-clé 1, Mot-clé 2}

% %--------------------
% % Change the layout and the style of the text of the "primary" abstract.
% % If your document is written in French, the primary abstract is in French,
% % otherwise it is in English.
%\Set{primaryabstractstyle}{\tiny}

% %--------------------
% % Change the layout and the style of the text of the "secondary" abstract.
% % If your document is written in French, the secondary abstract is in English,
% % otherwise it is in French.
%\Set{secondaryabstractstyle}{\tiny}

% %--------------------
% % Change the layout and the style of the text of the "primary" keywords.
% % If your document is written in French, the primary keywords are in French,
% % otherwise they are in English.
%\Set{primarykeywordstyle}{\tiny}

% %--------------------
% % Change the layout and the style of the text of the "secondary" keywords.
% % If your document is written in French, the secondary keywords are in English,
% % otherwise they are in French.
%\Set{secondarykeywordstyle}{\tiny}

% %--------------------
% % Change the specialty of the PhD thesis. Specialty must be selected from the official list of specialties from the doctoral school
%\Set{speciality}{Informatique}
%\Set{speciality}{Intelligence Artificielle}
%\Set{speciality}{Informatique de l'image}
%\Set{speciality}{Automatique}

% %--------------------
% % Change the institution
%\Set{universityname}{Universit\'e de Technologie de Belfort-Montb\'eliard}

% %--------------------
% % Clear the list of the laboratories
%\resetlaboratories

% %--------------------
% % Add the laboratory where the thesis was made
%\addlaboratory{Laboratoire Connaissance et Intelligence Artificielle Distribu\'ees}
%\setlaboratorylogo{bigciadlogo}

% %--------------------
% % The name of the university that is jointly delivering the Doctoral degree with UBFC
%\Set{jointuniversity}{University of Ngaoundéré in Cameroon}

% %--------------------
% % Clear the list of the partner/sponsor logos
%\resetpartners

% %--------------------
% % Add the logos of the partners or the sponsors on the front page
% %
% % CAUTION 1: At least, the logo of the University should appear (UTBM)
% %
%\addpartner[image options]{image name}

%\addpartner{utbm}

% %--------------------
% % Change the header and the foot of the pages.
% % You must include the package "fancyhdr" to
% % have access to these macros.
% % Left header
%\lhead{}
% % Center header
%\chead{}
% % Right header
%\rhead{}
% % Left footer
%\lfoot{}
% % Center footer
%\cfoot{}
% % Right footer
%\rfoot{}

% %--------------------
% Declare several theorems
\declareupmtheorem{mytheorem}{My Theorem}{List of my Theorems}{mythm}{mytheorem}{\textbf}

% %--------------------
% % Change the message on the backcover.
%\Set{backcovermessage}{%
%	Some text
%}

% %--------------------
% % Configure the acronyms' definitions
% % See the acro package documentation for details on the command
\usepackage{acro}
\acsetup{
  single = true,
  make-links = true,
}
% Declare an acronym
\DeclareAcronym{MAS}{
  short = MAS,
  long = {Multi-Agent System},
}

\begin{document}

% %--------------------
% % The following line does nothing until
% % the class option 'nofrontmatter' is given.
\frontmatter

% %--------------------
% % The following line permits to add a chapter for "acknowledgements"
% % at the beginning of the document. This chapter has not a chapter
% % number (using the "star-ed" version of \chapter) to prevent it to
% % be in the table of contents
\chapter*{Acknowledgements}

% %--------------------
% % Include a general table of contents
\tableofcontents

% %--------------------
% % Long summary in French
\chapter*{R\'esum\'e long}

\begin{upmcaution}
	You must write here a long summary of your PhD thesis in French language.
	According to the SPIM rules, the length of this long summary must be of minimum 3 pages for people who is not native-french speaker, and of minimum 20 pages for who is native-french speaker.
\end{upmcaution}


% %--------------------
% % Show acronyms
\printacronyms

% %--------------------
% % The content of the PhD thesis
\mainmatter

\part{Context and Issues}

\chapter{Introduction}

This is an acronym: \ac{MAS}.
This is the same acronym: \ac{MAS}.

\begin{researchquestion}[a name]
   Description of the research question.
\end{researchquestion}

\begin{objective}[a name]
   Description of the objective.
\end{objective}

\begin{contribution}[a name]
   Description of the contribution.
\end{contribution}

\section{Context}

This template describes some elements that can help you write your thesis.
A typical outline for a scientific thesis is also proposed.

\section{Thesis Objectives}

The main objective of your thesis can be highlighted using the environment below:

\begin{emphbox}
	Propose a model that does something!
\end{emphbox}

\section{Thesis Outline}

\chapter{State of the Art}

% Automatic Lettrine and Minitoc at the beginning of the chapter.
%
% The following macros are formatting the text at the beginning of a chapter according to the
% standard format.
%
% \chapterintro               See \chapterintrotosection.
%
% \chapterintro*              Similar to \chapterintro, except that the minitoc will be ignored.
%
% \chapterintrotosection      transform to lettrine the first letter that is following this macro
%                             until the next following \section macro.
%                             YOU MUST type the \section macro
%                             in the same file as the \chapterintrotosection macro.
%                             AND
%                             put a minitoc (if the minitoc package is included) just before
%                             the next following \section macro.
%
% \chapterintrotoinput        transform to lettrine the first letter that is following this macro
%                             until the next following \input macro.
%                             YOU MUST type the \input macro
%                             in the same file as the \chapterintrotoinput macro.
%                             AND
%                             put a minitoc (if the minitoc package is included) just before
%                             the next following \input macro.
%
% \chapterintrotoinclude      transform to lettrine the first letter that is following this macro
%                             until the next following \include macro.
%                             YOU MUST type the \include macro
%                             in the same file as the \chapterintrotoinclude macro.
%                             AND
%                             put a minitoc (if the minitoc package is included) just before
%                             the next following \include macro.

%\chapterintro*
\chapterintro
%\chapterintrotosection
%\chapterintrotoinput
%\chapterintrotoinclude

To help you write your thesis, several tools are described below.
Many other macros are available in the \LaTeX\ package set \texttt{tex-upmethodology}
on which the style of this thesis is based. Examples include environments for automatically creating subfigures and macros for defining unnumbered sections that appear in the table of contents.

\section{Propose a Definition}

Definition~\ref{def:athesis} illustrates the proposal of a definition.

\begin{definition}{A Thesis}{athesis}
Document presented to a university jury for obtaining a doctorate.
\end{definition}

\section{Include a Figure}

Including a figure is done using standard \LaTeX\ tools (environment \texttt{figure}, \texttt{{\textbackslash}includegraphics}, etc.).

We propose a macro to simplify the inclusion of a figure.

\begin{verbatim}
\mfigure[position]{options}{filename}{title}{labelid}
\end{verbatim}

This is equivalent to (note the addition of \texttt{fig:} as a prefix to the label):
\begin{verbatim}
\begin{figure}[position]
	\begin{center}
		\includegraphics[options]{filename}
		\label{fig:labelid}
		\caption{title}
	\end{center}
\end{figure}
\end{verbatim}

Referencing the figure can be done using the macros:
\begin{verbatim}
\figref{labelid}
\figpageref{labelid}
\end{verbatim}

\section{Include a Table}

Including a table is done using standard \LaTeX\ tools (environment \texttt{table}, environment \texttt{tabularx}, etc.).

We propose a macro to simplify the inclusion of a table.

\begin{verbatim}
\begin{mtable}[options]{width}{numberofcolumns}{columnspec}{title}{labelid}
	content
\end{mtable}
\end{verbatim}

This is equivalent to (note the addition of \texttt{tab:} as a prefix to the label):
\begin{verbatim}
\begin{table}[options]
	\begin{center}
		\begin{tabularx}{width}{columnspec}
			content
		\end{tabularx}
		\label{tab:labelid}
		\caption{title}
	\end{center}
\end{table}
\end{verbatim}

Referencing the table can be done using the macros:
\begin{verbatim}
\tabref{labelid}
\tabpageref{labelid}
\end{verbatim}

\subsection{Example 1}

Table \tabref{exampletable1} is an example of a table with 4 columns, with a title added at the top.
\begin{mtable}[ht]{.9\linewidth}{4}{|l|X|l|X|}{Table Title}{exampletable1}
	\tabulartitle{A title at the top}
	\tabularheader{Col1}{Col2}{Col3}{Col4}
	a & b & c & d \\
	\hline
	e & f & g & h \\
\end{mtable}

\subsection{Example 2}

Table \tabref{exampletable2} is an example of a table with 5 columns, with the table title also added at the top.
\begin{mtable}[ht]{.9\linewidth}{5}{|l|X|l|X|X|}{Table Title}{exampletable2}
	\captionastitle % Display the figure title at the top of the table
	\tabularheader{Col1}{Col2}{Col3}{Col4}{Col5}
	a & b & c & d & x \\
	\hline
	e & f & g & h & z \\
\end{mtable}

\section{Inline Enumeration}

You can enumerate elements in a paragraph: \begin{inlineenumeration}
\item element 1,
\item element 2,
\item element 3;
\end{inlineenumeration} and continue your text.

\section{Description}

The \texttt{description} environment provided by \LaTeX\ has been extended:
\begin{description}
\item[Element 1] Text 1
\item[Element 2] Text 2
\item[Element 3] Text 3
\end{description}

Omitting an item header is not a problem:
\begin{description}
\item[Element 1] Text 1
\item Text 2
\item[Element 3] Text 3
\end{description}

\section{Enumeration}

The \texttt{enumerate} environment provided by \LaTeX\ has been extended to combine the advantages of the \texttt{enumerate} and \texttt{description} environments in a single \LaTeX\ environment:
\begin{enumerate}
\item[Element 1] Text 1
\item[Element 2] Text 2
\item[Element 3] Text 3
\end{enumerate}

You can specify the type of enumeration by switching to Arabic numerals:
\begin{enumerate}[1]
\item[Element 1] Text 1
\item[Element 2] Text 2
\item[Element 3] Text 3
\end{enumerate}

Or in Roman numerals:
\begin{enumerate}[i]
\item[Element 1] Text 1
\item[Element 2] Text 2
\item[Element 3] Text 3
\end{enumerate}

Or in alphabetical numerals:
\begin{enumerate}[a]
\item[Element 1] Text 1
\item[Element 2] Text 2
\item[Element 3] Text 3
\end{enumerate}

Omitting an item header is not a problem:
\begin{enumerate}
\item[Element 1] Text 1
\item Text 2
\item[Element 3] Text 3
\end{enumerate}

\section{Format Text}

You can place text \textup{as a superscript}. You can place text \textdown{as a subscript}.

You can highlight \emph{text}, or highlight it \Emph{even more}.

You can format people's names uniformly, for example \makename{Stéphane}{Galland} (other macros are available).

\section{Mathematical Symbols}

\begin{itemize}
\item \R
\item \N
\item \Z
\item \Q
\item \C
\item $\powerset{a}$
\item $\sgn(a)$
\item $\min(a, b)$
\item $\max(a, b)$
\end{itemize}

\section{Theorems}

You can define your own environment to describe a theorem, lemma, etc.
This type of environment must be declared in the preamble of your document with the
macro \texttt{{\textbackslash}declareupmtheorem} (see the example in the preamble of
this template).

\begin{mytheorem}{Some Theorem}{thelabel}[This is my optional source]
	This is the description of this theorem.
\end{mytheorem}

At the end of your document, you can then add a chapter listing the theorems present in your document: \texttt{{\textbackslash}listofmytheorems}

\section{Conclusion}

%% Citation from the general bibliography
%\cite{key}

%% Citation from the PERSO bibliography
%\citePERSO{key}

\part{Contribution}

\chapter{Contribution}

\section{Introduction}

\section{Details of the Contribution}

\section{Conclusion}

\chapter{Implementation}

\section{Introduction}

\section{Presentation of the Implementation}

\section{Experimental Results}

\section{Conclusion}

\part{Conclusion}

\chapter{General Conclusion}

\section{Summary}

\section{Perspectives}

%%--------------------
%% Start the end of the thesis
\backmatter

%%--------------------
%% Bibliography

%% PERSONAL BIBLIOGRAPHY (use 'multibib')

%% Change the style of the PERSONAL bibliography
%\bibliographystylePERSO{phdthesisapa}

%% Add the chapter with the PERSONAL bibliography.
%% The name of the BibTeX file may be the same as
%% the one for the general bibliography.
%\bibliographyPERSO{biblio.bib}

%% Below, include a chapter for the GENERAL bibliography.
%% It is assumed that the standard BibTeX tool/approach
%% is used.

%% GENERAL BIBLIOGRAPHY

%% To cite one of your PERSONAL papers with the style
%% of the PERSONAL bibliography: \cite{key}

%% To force to show one of your PERSONAL papers into
%% the PERSONAL bibliography, even if not cited in the
%% text: \nocite{key}

%% The following line sets the style of
%% the GENERAL bibliography.
%% The following is an adaptation of "dinat.bst".
\bibliographystyle{spimphdthesis}

%% Link the GENERAL bibliography to a BibTeX file.
\bibliography{biblio.bib}

%%--------------------
%% List of figures and tables

%% Include a chapter with a list of all the figures.
%% In French typographic standards, this list must be at
%% the end of the document.
\listoffigures

%% Include a chapter with a list of all the tables.
%% In French typographic standards, this list must be at
%% the end of the document.
\listoftables

%%--------------------
%% Include a list of definitions
\listofdefinitions

%%--------------------
%% Include a list of algorithms
%\listofalgorithms

%%--------------------
%% Appendixes
\appendix
\part{Appendices}

\chapter{First Appendix Chapter}

\chapter{Second Appendix Chapter}

\end{document}

