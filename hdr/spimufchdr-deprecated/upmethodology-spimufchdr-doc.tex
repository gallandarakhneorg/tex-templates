\documentclass[english]{spimufchdr}

% The TeX code is entering with UTF8
% character encoding (Linux and MacOS standards)
\usepackage[utf8]{inputenc}

\usepackage{verbatim}

\makeatletter
\upm@extension@savealllangtrue

\gdef\upmextensiondoccolorlist{}
\gdef\upmextensiondocvaluelist{}

\global\let\upm@extensiondoc@old@definecolor\definecolor
\global\let\upm@extensiondoc@old@setpdfcolor\setpdfcolor
\global\let\upm@extensiondoc@old@setvalue\upm@extension@Set

\renewcommand{\definecolor}[3]{%
	\upm@extensiondoc@old@definecolor{#1}{#2}{#3}%
	\protected@xdef\upmextensiondoccolorlist{\upmextensiondoccolorlist%
		\protect\upm@protect{#1} & #2 & #3 & \protect\textcolor{#1}{\protect\upm@protect{#1}} \protect\\%
	}%
}

\renewcommand{\setpdfcolor}[1]{%
	\upm@extensiondoc@old@setpdfcolor{#1}%
	\upm@extensiondoc@old@definecolor{my__pdf__color}{rgb}{#1}%
	\protected@xdef\upmextensiondoccolorlist{\upmextensiondoccolorlist%
		``PDFCOLOR'' & rgb & #1 & \protect\textcolor{my__pdf__color}{PDF color} \protect\\%
	}%
}

\let\MyProtect\upm@protect
\newcommand{\MyGet}[2]{%
	\ifthenelse{\equal{#2}{copyright}}{-}{%
	\ifthenelse{\equal{#2}{trademarks}}{-}{%
	\ifthenelse{\equal{#2}{cfrontpage}}{-}{%
	\ifthenelse{\equal{#2}{backpage}}{-}{%
	\ifthenelse{\equal{#2}{abstract}}{-}{%
	\ifthenelse{\equal{#2}{keywords}}{-}{%
	\ifthenelse{\equal{#2}{logo}}{-}{%
	\ifthenelse{\equal{#2}{smalllogo}}{-}{%
	\ifthenelse{\equal{#2}{frontillustration}}{-}{%
	\ifthenelse{\equal{#2}{headerillustration}}{-}{%
	\GetLang{#2}{#1}%
	}}}}}}}}}}%
}

\renewcommand{\Set}[3][default]{%
	\ifthenelse{\equal{#1}{default}}{%
		\upm@extensiondoc@old@setvalue{#2}{#3}%
		\protected@xdef\upmextensiondocvaluelist{\upmextensiondocvaluelist%
			\protect\upm@protect{#2} & \protect\textit{all} & \protect\MyGet{\upmcurrentlang}{#2} \protect\\%
		}%
	}{%
		\upm@extensiondoc@old@setvalue[#1]{#2}{#3}%
		\protected@xdef\upmextensiondocvaluelist{\upmextensiondocvaluelist%
			\protect\upm@protect{#2} & \protect\upm@protect{#1} & \protect\MyGet{#1}{#2} \protect\\%
		}%
	}%
}

\let\VERversion\upm@package@version@ver
\let\VERfmt\upm@package@fmt@ver
\let\VERdoc\upm@package@doc@ver
\let\VERfp\upm@package@fp@ver
\let\VERbp\upm@package@bp@ver
\let\VERext\upm@package@ext@ver
\let\VERtask\upm@package@task@ver
\let\VERdocclazz\upm@package@docclazz@ver
\let\VERcode\upm@package@code@ver
\let\VERcommon\upm@package@private@doctype@ver
\makeatother

\declarehdr[Official Documentation]{SPIM-UFC HDR Style Extension for tex-upmethodology}{17 septembre 2012}

\addauthor[stephane.galland@UFC.fr]{St\'ephane}{Galland}

\addjury{Super}{Man}{Reviewer}{Professor at the University of Gotham City}
\addjury{Bat}{Man}{Supervisor}{Professor at the University of Gotham City}

\hdrabstract[english]{This document describes the style extension for \texttt{tex-upmethodology} which is providing the standard document layout and colors of the HDR from the SPIM doctoral school and the Universit\'e de Franche-Comt\'e, France.}

\hdrabstract[french]{Ce document d\'ecrit l'extension de \texttt{tex-upmethodology} d\'edi\'ee au standard graphique des HDR de l'\'Ecole Doctole SPIM et de l'Universit\'e de Franche-Comt\'e, France.}

\begin{document}

\mainmatter

\chapter{Introduction}

This document describes the style extension for \texttt{tex-upmethodology} which is providing the standard document layout and colors of the HDR from the Universit\'e de Franche-Comt\'e, France.

This document contains only the values and the macros provided by the extensions.

\section{Usage}

\textbf{THE ACCOMPANYING EXTENSION IS PROVIDED UNDER THE TERMS OF THIS LICENSE AGREEMENT. ANY USE, REPRODUCTION OR DISTRIBUTION OF THE EXTENSION CONSTITUTES RECIPIENT'S ACCEPTANCE OF THIS AGREEMENT.}

\textbf{YOU ARE GRANTED TO USE, REPRODUCE OR DISTRIBUTE THE spimufchdr STYLE EXTENTION FOR TEX-UPMETHODOLOGY LATEX PACKAGE IF AND ONLY IF: A) YOU ARE A REGISTERED MEMBER OF THE UNIVERSIT\'E DE FRANCHE-COMT\'E, OR B) YOU ARE OFFICIALLY AND EXPLICITELY AUTHORIZED TO USE THIS EXTENSION BY A REPRESENTATIVE OF THE UNIVERSIT\'E DE FRANCHE-COMT\'E.}

\section{Requirements}

\texttt{tex-upmethdology-spimufchdr} requires the core packages of \texttt{tex-upmethodology}.

\section{Installation}

Copy \texttt{tex-upmethdology-spimufchdr} configuration file (\texttt{.cfg}) and all related pictures in your system-wide or local user \texttt{texmf}.

\chapter{Provided Macros}

\texttt{tex-upmethdology-spimufchdr} providesthe following macros:
\begin{itemize}
\item \texttt{{\textbackslash}declarehdr[subtitle]\{title\}\{defense date\}} \\
	Declare the major information about the HDR:
	\begin{itemize}
	\item \texttt{title} is the title of the HDR that may appear on the front pages.
	\item \texttt{subtitle} is an optional subtitle for the HDR that may appear on the front pages.
	\item \texttt{defense data} is the data at which the HDR is defended.
	\item \texttt{PhD ID} is the identifier or the number of the HDR in the University's library.
	\end{itemize}
\item \texttt{{\textbackslash}addjury[email]\{first name\}\{last name\}\{role\}\{position\}} \\
	Add a member to the jury to which the HDR was presented:
	\begin{itemize}
	\item \texttt{first name} is the first name of the member of the jury.
	\item \texttt{last name} is the last name of the member of the jury.
	\item \texttt{role} is the role of the member in the jury, eg. in French ``Rapporteur'', ``Examinateur'', ``Directeur de Th\`ese''.
	\item \texttt{position} is the position and the University of the member.
	\end{itemize}
\item \texttt{{\textbackslash}thejurytab} \\
	Is expanded to the tabular that contains the jury members.
\item \texttt{{\textbackslash}hdrpreparedin\{institution name\}} \\
	Specify the name of the institution in which the HDR was prepared.
\item \texttt{{\textbackslash}thehdrpreparedin} \\
	Is expanded to the name of the institution in which the HDR in prepared.
\item \texttt{{\textbackslash}addpartner[options]\{partner picture\}} \\
	Add a picture for a partner on the front page. The options are passed to \texttt{{\textbackslash}includegraphics}.
\item \texttt{{\textbackslash}thepartnerlist} \\
	Is expanded to the pictures of the partners.
\item \texttt{{\textbackslash}hdrabstract[lang]\{text\}} \\
	Set the abstract of the HDR for the given language.
	\begin{itemize}
	\item \texttt{lang} must be the language in which the abstract is written. It must be one of ``french'' or ``english''.
	\item \texttt{text} is the text of the abstract.
	\end{itemize}
\item \texttt{{\textbackslash}hdrmainabstract} \\
	Is expanded to the text of the abstract for the primary language.
\item \texttt{{\textbackslash}hdrminorabstract} \\
	Is expanded to the text of the abstract for the secondary language.
\item \texttt{{\textbackslash}hdrkeywords[lang]\{text\}} \\
	Set the keywords of the HDR for the given language.
	\begin{itemize}
	\item \texttt{lang} must be the language in which the keywords is written. It must be one of ``french'' or ``english''.
	\item \texttt{text} is the text of the keywords.
	\end{itemize}
\item \texttt{{\textbackslash}hdrmainkeywords} \\
	Is expanded to the text of the keywords for the primary language.
\item \texttt{{\textbackslash}hdrminorkeywords} \\
	Is expanded to the text of the keywords for the secondary language.
\item \texttt{{\textbackslash}eme} \\
	Is expanded to ``\eme''.
\item \texttt{{\textbackslash}parbreak} \\
	Add a vertical space that is corresponding to a paragraph separator.
\item \texttt{{\textbackslash}hdrunderlineauthor\{name key\}\{name\}} \\
	Output the given name, and underline it if the given \texttt{name key} is corresponding to the name of the author of the HDR.
\item \texttt{{\textbackslash}declarebiblio[options]\{key\}\{label\}\{bibtex file\}} \\
	Declare a family of publications. The family is identified by the given \texttt{key}. The given \texttt{label} is used as the section's name of this family in the publication's list. The publications in this family are read from the given Bib\TeX\ file. The options are: 
	\begin{itemize}
	\item \texttt{style=bst style} specifies the name of the BST style to use for the family (default: \texttt{hdr}).
	\item \texttt{header=text} is the text to put in the header of the pages for the family.
	\end{itemize}
\item \texttt{{\textbackslash}addbiblio\ensuremath{\theta}\{key\}} \\
	Add a publication in a family of publications identified by $\theta$. The given \texttt{key} is the key in the Bib\TeX\ file.
\item \texttt{{\textbackslash}numberof\ensuremath{\theta}} \\
	Is expanded with the number of publications in the family $\theta$.
\item \texttt{{\textbackslash}hdrpublicationlist} \\
	Is expanded to the list of the publications of the author of this HDR (see \texttt{{\textbackslash}declarebiblio}).
\item \texttt{{\textbackslash}bibentry\{key\}} \\
	Is expanded with the details of the entry identified by \texttt{key} in the Bib\TeX\ file.
\item \texttt{{\textbackslash}projectdesc\{title\}\{budget\}\{start\}\{end\}\{partners\}\{goals\}\{roles\}} \\
	Declare a project with the given \texttt{title} and \texttt{budget}. The project starts and ends at the given dates. The partners and the goals of the project are specified. The roles of the authors are given by \texttt{roles}.
\item \texttt{{\textbackslash}projectdescstar\{title\}\{budget\}\{start\}\{end\}\{partners\}\{goals\}\{roles\}} \\
	Declare a project with the given \texttt{title} and \texttt{budget}. The project starts and ends at the given dates. The partners and the goals of the project are specified. The single role of the authors is given by \texttt{role}.
\item \texttt{{\textbackslash}projectdescnat\{title\}\{start\}\{end\}\{funder\}\{partners\}\{goals\}\{roles\}} \\
	Declare a project funded by the \texttt{funder} with the given \texttt{title}. The project starts and ends at the given dates. The partners and the goals of the project are specified. The single role of the authors is given by \texttt{role}.
\end{itemize}

\chapter{Provided Colors}

\texttt{tex-upmethdology-spimufchdr} provides the following colors. The colors are expressed according to the standard \texttt{color.sty} specifications.

\begin{mtabular}{4}{XccX}
\tabulartitle{Provided Colors}
\tabularheader{Name}{Type}{Value}{Example}
\upmextensiondoccolorlist
\end{mtabular}

\chapter{Provided Values}

\texttt{tex-upmethdology-spimufchdr} provides the following values. The values are accessible with \texttt{{\textbackslash}Get} macro.

\begin{mtabular}{3}{XcX}
\tabulartitle{Provided Values}
\tabularheader{Name}{Language}{Content}
\upmextensiondocvaluelist
\end{mtabular}

\chapter{Legal}

\section{Copyright of the Extension's Source Code}

\Get{copyright}

\Get{trademarks}

\section{Copyright of Materials}

Copyright \copyright 2012--13 \'Ecole Doctorale Science Pour L'Ing\'enieur et Micro\'electronique. All rights reserved.

All SPIM logos are the property of the \'Ecole Doctorale Science Pour L'Ing\'enieur et Micro\'electronique, France.

All UFC logos are the property of the Universit\'e de Franche-Comt\'e, France.

You are not granted to use, modify or distribute this package without the explicit authorization of the Universit\'e de Franche-Comt\'e.

\chapter{Template of a HDR}

\verbatiminput{example}

\chapter{Generator}

This document was generated with the following versions of the UP-methodology packages:
\begin{itemize}
\item upmethodology-version: \VERversion
\item upmethdology-fmt: \VERfmt
\item upmethdology-document: \VERdoc
\item upmethdology-frontpage: \VERfp
\item upmethdology-backpage: \VERbp
\item upmethdology-extension: \VERext
\item upmethdology-task: \VERtask
\item upmethdology-document (class): \VERdocclazz
\item upmethdology-code: \VERcode
\item Private API: \VERcommon
\end{itemize}

\end{document}
