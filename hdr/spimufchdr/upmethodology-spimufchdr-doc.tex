\documentclass[article,english]{upmethodology-document}

\usepackage{verbatim}

\makeatletter
\upm@extension@savealllangtrue

\gdef\upmextensiondoccolorlist{}
\gdef\upmextensiondocvaluelist{}

\global\let\upm@extensiondoc@old@definecolor\definecolor
\global\let\upm@extensiondoc@old@setpdfcolor\setpdfcolor
\global\let\upm@extensiondoc@old@setvalue\upm@extension@Set

\renewcommand{\definecolor}[3]{%
	\upm@extensiondoc@old@definecolor{#1}{#2}{#3}%
	\protected@xdef\upmextensiondoccolorlist{\upmextensiondoccolorlist%
		\protect\upm@protect{#1} & #2 & #3 & \protect\textcolor{#1}{\protect\upm@protect{#1}} \protect\\%
	}%
}

\renewcommand{\setpdfcolor}[1]{%
	\upm@extensiondoc@old@setpdfcolor{#1}%
	\upm@extensiondoc@old@definecolor{my__pdf__color}{rgb}{#1}%
	\protected@xdef\upmextensiondoccolorlist{\upmextensiondoccolorlist%
		``PDFCOLOR'' & rgb & #1 & \protect\textcolor{my__pdf__color}{PDF color} \protect\\%
	}%
}

\let\MyProtect\upm@protect
\newcommand{\MyGet}[2]{%
	\ifthenelse{\equal{#2}{copyright}}{-}{%
	\ifthenelse{\equal{#2}{trademarks}}{-}{%
	\ifthenelse{\equal{#2}{cfrontpage}}{-}{%
	\ifthenelse{\equal{#2}{backpage}}{-}{%
	\ifthenelse{\equal{#2}{abstract}}{-}{%
	\ifthenelse{\equal{#2}{keywords}}{-}{%
	\ifthenelse{\equal{#2}{logo}}{-}{%
	\ifthenelse{\equal{#2}{smalllogo}}{-}{%
	\ifthenelse{\equal{#2}{frontillustration}}{-}{%
	\ifthenelse{\equal{#2}{headerillustration}}{-}{%
	\GetLang{#2}{#1}%
	}}}}}}}}}}%
}

\renewcommand{\Set}[3][default]{%
	\ifthenelse{\equal{#1}{default}}{%
		\upm@extensiondoc@old@setvalue{#2}{#3}%
		\protected@xdef\upmextensiondocvaluelist{\upmextensiondocvaluelist%
			\protect\upm@protect{#2} & \protect\textit{all} & \protect\MyGet{\upmcurrentlang}{#2} \protect\\%
		}%
	}{%
		\upm@extensiondoc@old@setvalue[#1]{#2}{#3}%
		\protected@xdef\upmextensiondocvaluelist{\upmextensiondocvaluelist%
			\protect\upm@protect{#2} & \protect\upm@protect{#1} & \protect\MyGet{#1}{#2} \protect\\%
		}%
	}%
}

\let\VERversion\upm@package@version@ver
\let\VERfmt\upm@package@fmt@ver
\let\VERdoc\upm@package@doc@ver
\let\VERfp\upm@package@fp@ver
\let\VERbp\upm@package@bp@ver
\let\VERext\upm@package@ext@ver
\let\VERtask\upm@package@task@ver
\let\VERdocclazz\upm@package@docclazz@ver
\let\VERcode\upm@package@code@ver
\let\VERcommon\upm@package@private@doctype@ver
\makeatother

\UseExtension{spimufchdr}

\declarehdr[Official Documentation]{SPIM-UFC HDR Style Extension for tex-upmethodology}{17 septembre 2012}

\addauthor[stephane.galland@UFC.fr]{St\'ephane}{Galland}

\addjury{Super}{Man}{Reviewer}{Professor at the University of Gotham City}
\addjury{Bat}{Man}{Supervisor}{Professor at the University of Gotham City}

\hdrabstract[english]{This document describes the style extension for \texttt{tex-upmethodology} which is providing the standard document layout and colors of the HDR from the SPIM doctoral school and the Universit\'e de Franche-Comt\'e, France.}

\hdrabstract[french]{Ce document d\'ecrit l'extension de \texttt{tex-upmethodology} d\'edi\'ee au standard graphique des HDR de l'\'Ecole Doctole SPIM et de l'Universit\'e de Franche-Comt\'e, France.}

\begin{document}

\section{Introduction}

This document describes the style extension for \texttt{tex-upmethodology} which is providing the standard document layout and colors of the HDR from the Universit\'e de Franche-Comt\'e, France.

This document contains only the values and the macros provided by the extensions.

\section{Usage}

\textbf{THE ACCOMPANYING EXTENSION IS PROVIDED UNDER THE TERMS OF THIS LICENSE AGREEMENT. ANY USE, REPRODUCTION OR DISTRIBUTION OF THE EXTENSION CONSTITUTES RECIPIENT'S ACCEPTANCE OF THIS AGREEMENT.}

\textbf{YOU ARE GRANTED TO USE, REPRODUCE OR DISTRIBUTE THE spimufchdr STYLE EXTENTION FOR TEX-UPMETHODOLOGY LATEX PACKAGE IF AND ONLY IF: A) YOU ARE A REGISTERED MEMBER OF THE UNIVERSIT\'E DE FRANCHE-COMT\'E, OR B) YOU ARE OFFICIALLY AND EXPLICITELY AUTHORIZED TO USE THIS EXTENSION BY A REPRESENTATIVE OF THE UNIVERSIT\'E DE FRANCHE-COMT\'E.}

\section{Requirements}

\texttt{tex-upmethdology-spimufchdr} requires the core packages of \texttt{tex-upmethodology}.

\section{Installation}

Copy \texttt{tex-upmethdology-spimufchdr} configuration file (\texttt{.cfg}) and all related pictures in your system-wide or local user \texttt{texmf}.

\section{Provided Macros}

\texttt{tex-upmethdology-spimufchdr} providesthe following macros:
\begin{itemize}
\item \texttt{{\textbackslash}declarehdr[subtitle]\{title\}\{defense date\}} \\
	Declare the major information about the HDR:
	\begin{itemize}
	\item \texttt{title} is the title of the HDR that may appear on the front pages.
	\item \texttt{subtitle} is an optional subtitle for the HDR that may appear on the front pages.
	\item \texttt{defense data} is the data at which the HDR is defended.
	\item \texttt{PhD ID} is the identifier or the number of the HDR in the University's library.
	\end{itemize}
\item \texttt{{\textbackslash}addjury[email]\{first name\}\{last name\}\{role\}\{position\}} \\
	Add a member to the jury to which the HDR was presented:
	\begin{itemize}
	\item \texttt{first name} is the first name of the member of the jury.
	\item \texttt{last name} is the last name of the member of the jury.
	\item \texttt{role} is the role of the member in the jury, eg. in French ``Rapporteur'', ``Examinateur'', ``Directeur de Th\`ese''.
	\item \texttt{position} is the position and the University of the member.
	\end{itemize}
\item \texttt{{\textbackslash}hdrabstract[lang]\{text\}} \\
	Set the French abstract for the HDR.
	\begin{itemize}
	\item \texttt{lang} must be the language in which the abstract is written. It must be one of ``french'' or ``english''.
	\item \texttt{text} is the text of the abstract.
	\end{itemize}
\end{itemize}

\section{Provided Colors}

\texttt{tex-upmethdology-spimufchdr} provides the following colors. The colors are expressed according to the standard \texttt{color.sty} specifications.

\begin{mtabular}{4}{XccX}
\tabulartitle{Provided Colors}
\tabularheader{Name}{Type}{Value}{Example}
\upmextensiondoccolorlist
\end{mtabular}

\section{Provided Values}

\texttt{tex-upmethdology-spimufchdr} provides the following values. The values are accessible with \texttt{{\textbackslash}Get} macro.

\begin{mtabular}{3}{XcX}
\tabulartitle{Provided Values}
\tabularheader{Name}{Language}{Content}
%\upmextensiondocvaluelist
\end{mtabular}

\section{Copyright of the Extension's Source Code}

\Get{copyright}

\Get{trademarks}

\section{Copyright of Materials}

Copyright \copyright 2012--13 \'Ecole Doctorale Science Pour L'Ing\'enieur et Micro\'electronique. All rights reserved.

All SPIM logos are the property of the \'Ecole Doctorale Science Pour L'Ing\'enieur et Micro\'electronique, France.

All UFC logos are the property of the Universit\'e de Franche-Comt\'e, France.

You are not granted to use, modify or distribute this package without the explicit authorization of the Universit\'e de Franche-Comt\'e.

\section{Template of a HDR}

\verbatiminput{example}

\section{Generator}

This document was generated with the following versions of the UP-methodology packages:
\begin{itemize}
\item upmethodology-version: \VERversion
\item upmethdology-fmt: \VERfmt
\item upmethdology-document: \VERdoc
\item upmethdology-frontpage: \VERfp
\item upmethdology-backpage: \VERbp
\item upmethdology-extension: \VERext
\item upmethdology-task: \VERtask
\item upmethdology-document (class): \VERdocclazz
\item upmethdology-code: \VERcode
\item Private API: \VERcommon
\end{itemize}

\end{document}
